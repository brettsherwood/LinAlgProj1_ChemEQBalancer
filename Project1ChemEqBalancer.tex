\documentclass[10pt]{article}

\usepackage[utf8]{inputenc} % set input encoding (not needed with XeLaTeX)
\usepackage[document]{ragged2e} % used for text alignment

\usepackage{geometry} % to change the page dimensions
\geometry{a4paper} % or letterpaper (US) or a5paper or\dots.
\geometry{margin=.5in} % change the margins to .5 inches all round

%\setlength{\parskip}{1em} % sets width between paragraphs

\usepackage{array} % for better arrays (eg matrices) in maths
\usepackage{paralist} % very flexible & customisable lists (eg. enumerate/itemize, etc.)
\usepackage{harpoon} % use to add ``harpoons'' over letters for vector notation
\usepackage{amssymb} % symbolsss

\usepackage{multicol}	% for multi columns
\setlength{\columnsep}{.5cm} % change separation width of columns

\usepackage{soul} % for underlines - \ul
\setul{1pt}{.4pt} % 1pt below text, .4pt width

\usepackage{amsmath} % all the math things - \text and \quad

\usepackage{listings}
\usepackage{xcolor}
 
\definecolor{codegreen}{rgb}{0,0.6,0}
\definecolor{codegray}{rgb}{0.5,0.5,0.5}
\definecolor{codepurple}{rgb}{0.58,0,0.82}
\definecolor{backcolour}{rgb}{0.95,0.95,0.92}
 
\lstdefinestyle{mystyle}{
  backgroundcolor=\color{backcolour},  
  commentstyle=\color{red},
  keywordstyle=\bf\color{magenta},
  numberstyle=\tiny\color{codegray},
  stringstyle=\color{codepurple},
  basicstyle=\sffamily\small,
  breakatwhitespace=false,     
  breaklines=true,         
  captionpos=b,          
  keepspaces=true,         
  numbers=left,          
  numbersep=5pt,         
  showspaces=false,        
  showstringspaces=false,
  showtabs=false,         
  tabsize=2,
  frame=single,
  rulecolor=\color{black}
}
\lstset{style=mystyle}

% backgroundcolor - colour for the background. External color or xcolor package needed.
% commentstyle - style of comments in source language.
% basicstyle - font size/family/etc. for source (e.g. basicstyle=\ttfamily\small)
% keywordstyle - style of keywords in source language (e.g. keywordstyle=\color{red})
% numberstyle - style used for line-numbers
% numbersep - distance of line-numbers from the code
% stringstyle - style of strings in source language
% showspaces - emphasize spaces in code (true/false)
% showstringspaces - emphasize spaces in strings (true/false)
% showtabs - emphasize tabulators in code (true/false)
% numbers - position of line numbers (left/right/none, i.e. no line numbers)
% prebreak - displaying mark on the end of breaking line (e.g. prebreak=\raisebox{0ex}[0ex][0ex]{\ensuremath{\hookleftarrow}})
% captionpos - position of caption (t/b)
% frame - showing frame outside code (none/leftline/topline/bottomline/lines/single/shadowbox)
% breakwhitespace - sets if automatic breaks should only happen at whitespaces
% breaklines - automatic line-breaking
% keepspaces - keep spaces in the code, useful for indetation
% tabsize - default tabsize
% escapeinside - specify characters to escape from source code to LATEX (e.g. escapeinside={\%*}{*)})
% rulecolor - Specify the colour of the frame-box

\begin{document}

\begin{center}
{\huge \bf Balancing Chemical Equations} \\
\vspace*{.5cm}
{\large Brett Webb \\
Red Rocks Community College \\
MAT 225-002 Linear Algebra \\
Professor: Adam Forland \\
\vspace*{.5cm}
October $4^{th}$, 2019} 
\end{center}

\section*{The Problem}
Our software development team has been asked to write a piece of software for a local highschool chemistry class.
The intention is to allow the students to enter in a chemical equation and the program will output the balanced 
equation (if possible). In addition, the software should include code comments to express how the code is working
as well as markdown/latex comments to express how the math (and chemistry) is working.

\section{Introduction}
As descrbed above, our goal was to develop a Python program that utilizes Linear Algebra to balance a given chemical
equation. Chemical equations describe the quantities of substances consumed and produced by chemcial reactions. The 
Law of Conservation of Mass states that no atoms can be created nor destroyed in a chmeical reaction. This means that
the number of atoms of the reactants (the left side of the equation) must be equal to the number of atoms of the
products(the right side). So in order to balance an equation, coefficients must be added to the molecules of the equation.
In the example and code to follow, you will see the process of how this is done by hand using matrix operations, and then
by the program that was developed.

%Written introduction and the results. Just like 5-6 sentences about what this paper is going to show and what happened.
%Explain the process and what not.

\section{Example}

For this example, we will use the following chemical equation,
\[
(x_{1})AgNO_{3} + (x_{2})MgCl_{2} \to (x_{3})AgCl + (x_{4})Mg(NO_{3})_{2}
\]
Notice the coefficients that are in front of the molecules, these are placeholders just to represent the variables that
we will be solving for in the following calculations. Keep in mind, these will always be integer numbers. \\
\hfill

First, we must convert the reactants and products into vectors,
\[
AgNO_{3} = \left[ \begin{array}{c} 1 \\ 1 \\ 3 \\ 0 \\ 0 \\ \end{array} \right], 
MgCl_{2} = \left[ \begin{array}{c} 0 \\ 0 \\ 0 \\ 1 \\ 2 \\ \end{array} \right],
AgCl = \left[ \begin{array}{c} 1 \\ 0 \\ 0 \\ 0 \\ 1 \\ \end{array} \right],
Mg(NO_{3})_{2} = \left[ \begin{array}{c} 0 \\ 2 \\ 6 \\ 1 \\ 0 \\ \end{array} \right]
\hspace{1cm} \left[ \begin{array}{c} Ag \\ N \\ O \\ Mg \\ Cl \\ \end{array} \right]
\]
These vectors each describe the numbers of atoms per molecule for each type of element present in the reaction. On the right
you can see a visual representation of what element each row represents. Since the first reactant has three different atoms
($Ag$, $N$, and $O$) the corresponding number of atoms in that molecule in placed in the vector in their corresponding
positions. Notice the last vector, the outside subscript of 2 represents that there are two molecules of $NO_{3}$ which means
there are a total of 2 $N$ atoms and 6 $O$ atoms.

\begin{multicols}{2}
Next, we replace the molecules in our equation with their given vectors and replace the $\to$ arrow with an equals sign
because we are trying to solve for the coefficiants that must satisy the equation.
\[
x_{1} \left[ \begin{array}{c} 1 \\ 1 \\ 3 \\ 0 \\ 0 \\ \end{array} \right] + 
x_{2} \left[ \begin{array}{c} 0 \\ 0 \\ 0 \\ 1 \\ 2 \\ \end{array} \right] = 
x_{3} \left[ \begin{array}{c} 1 \\ 0 \\ 0 \\ 0 \\ 1 \\ \end{array} \right] +
x_{4} \left[ \begin{array}{c} 0 \\ 2 \\ 6 \\ 1 \\ 0 \\ \end{array} \right]
\]

We then move all of the terms to left side of the equation to set the equation equal to 0. In this case, since we are using
vectors, we use the zero vector, $\overrightharp{0}$. This will change the signs of the vectors from the right side of the equation.
\[
x_{1} \left[ \begin{array}{c} 1 \\ 1 \\ 3 \\ 0 \\ 0 \\ \end{array} \right] + 
x_{2} \left[ \begin{array}{c} 0 \\ 0 \\ 0 \\ 1 \\ 2 \\ \end{array} \right] + 
x_{3} \left[ \begin{array}{c} -1 \\ 0 \\ 0 \\ 0 \\ -1 \\ \end{array} \right] +
x_{4} \left[ \begin{array}{c} 0 \\ -2 \\ -6 \\ -1 \\ 0 \\ \end{array} \right] =
\left[ \begin{array}{c} 0 \\ 0 \\ 0 \\ 0 \\ 0 \\ \end{array} \right]
\]

From here, we can create the augmented matrix of the system in which our vectors become the columns of the matrix and
the solution to our matrix, $\overrightharp{0}$, is separated by a vertical line.
\[
\left[
\begin{array}{cccc|c}
1 & 0 & -1 & 0 & 0 \\
1 & 0 & 0 & -2 & 0 \\
3 & 0 & 0 & -6 & 0 \\
0 & 1 & 0 & -1 & 0 \\
0 & 2 & -1 & 0 & 0 \\
\end{array}
\right]
\]

We then apply row reduce the matrix using row operations.
\[ 
\implies
\left[
\begin{array}{cccc|c}
1 & 0 & -1 & 0 & 0 \\
1 & 0 & 0 & -2 & 0 \\
0 & 0 & 0 & 0 & 0 \\
0 & 1 & 0 & -1 & 0 \\
0 & 2 &-1 & 0 & 0 \\
\end{array}
\right]
\]
\[
\implies
\left[
\begin{array}{cccc|c}
1 & 0 & -1 & 0 & 0 \\
0 & 0 & 1 & -2 & 0 \\
0 & 0 & 0 & 0 & 0 \\
0 & 1 & 0 & -1 & 0 \\
0 & 0 & -1 & 2 & 0 \\
\end{array}
\right]
\]
\[
\implies
\left[ 
\begin{array}{cccc|c}
1 & 0 & -1 & 0 & 0 \\
0 & 1 & 0 & -1 & 0 \\
0 & 0 & 1 & -2 & 0 \\
0 & 0 & 0 & 0 & 0 \\
0 & 0 & 0 & 0 & 0 \\
\end{array}
\right]
\]

After row reduction we are left with our matrix in ``Row Reduced Echelon Form'' (RREF).
\[
\left[
\begin{array}{cccc|c}
1 & 0 & 0 & -2 & 0 \\
0 & 1 & 0 & -1 & 0 \\
0 & 0 & 1 & -2 & 0 \\
0 & 0 & 0 & 0 & 0 \\
0 & 0 & 0 & 0 & 0 \\
\end{array}
\right]
\]

The solution to our matrix is then shown as,
\[
x = 
\left\{
\begin{array}{l}
x_{1} = 2x_{4} \\
x_{2} = x_{4} \\
x_{3} = 2x_{4} \\
x_{4}\ is\ free \\
\end{array}
\right.
\]

Since $x_{4}$ is free, we can let it be equal to any number and it will still be a solution to our matrix. When balancing
equations, it is best to keep coefficients as simplified as possible. So to do that, we let $x_{4} = 1$.
\[
x = 
\left\{
\begin{array}{l}
x_{1} = 2 \\
x_{2} = 1 \\
x_{3} = 2 \\
x_{4} = 1 \\
\end{array}
\right.
\]

These are the the coefficants to our chemcial equation. So, the balanced equation is
\[
2AgNO_{3} + MgCl_{2} \to 2AgCl + Mg(NO_{3})_{2}
\]
\end{multicols}

\section{The Code}
This code was developed through a collabaration effort between Jaden Adams, Connor Denney, Kevin Evers, Holly Hammons,
Alex Langfield, Everett Oklar, Brett Webb, and Sophia Wyss with the supervision of our professor Adam Forland.

This code will successfully take an unbalanced chemcial equation, convert it into a matrix, row reduce the matrix, solve for free variables, and find the solution, giving us the coefficients for the balanced chemical equation.
\begin{lstlisting}[language = Python]
from fractions import Fraction
import math
from sympy import *


# Takes a string and a delimiter character is an input and splits the string at the specified delimiter. Outputs the string as a tuple split at the delimiter.
def split(string, delimiter):	
	output = [[]]				# Creates a tuple that will be the output of the function
	counter = 0					# Used to count how many times we have split our string

	for i in range(len(string)):			# Loop ``the number of characters in string'' times
		if string[i] == delimiter:			# Tests if the current character is the delimiter
			counter += 1									# Increment the counter
			output.append([])							# Add a new string to the tuple
		else:														# If the current character is not the delimiter
			output[counter] += string[i]	# Add character to current string
	return(output)										# Output the list of strings


# Determines if an item is in a list and outputs the location of the item.
def existsin(thelist, item):
	for i in range(len(thelist)): 		# Loop ``the number of entries in thelist'' times
		if thelist[i] == item: 					# Tests if the current item is the item we're looking for
			return [True, i]							# Return True and location of the item.
	return [False, 0]									# Returns False if the item is not in the current list.

# Turns a compund string into a vector. inp is the compund string input; thiscompound is an empty vector of the same length in which we will store our resulting vector; elements is an empty list to keep track of what elements are in the equation. Outputs an updated elements list and the compound vector, thiscompound.
def vectorizeCompound(inp, thiscompound, elements):
	for i in range(len(inp)):				# Loop ``the number of characters in the string inp'' times
		if inp[i].isnumeric() == False:				# Tests if the current character is not a number 
			if i == 0:				# Tests if this is the first character in the element 
				elements.append(inp[i])				# Add character to elements list
				continue					# Skip the rest of this iteration and go to the next one
			if inp[i-1].isnumeric() == False:		# Tests if the last character was also not a number
				elements[len(elements)-1] += inp[i]				# Add this character to the existing character, rather than making a new item for that character
			else:
				elements.append(inp[i])		# Else if it's just a lone letter so far, add it to the list
    
		else:		# Else if the current character is a number
			if len(elements) == 0:		# If there are no elements yet in the list
				continue				# Something went wrong, so just keep going
			if inp[i-1].isnumeric():				# If the last character was also a number
				thiscompound[Existence[1]] = inp[i-1] + inp[i]				# Add this character to the existing number ** (concatination of strings, not addition)
				continue				# Continue to next iteration to avoid redefining the entry 
			NewElement = elements.pop(len(elements)-1)		# Remove the last element from the list
			Existence = existsin(elements, NewElement)		# Check to see if this element has already been counted
			if Existence[0] == False:		# If element has not been counted
				elements.append(NewElement)			# Add this element to the elements list
				thiscompound[len(elements)-1] = inp[i]		# Add the current number to the vector
				Existence[1] = len(elements)-1	# For the case where you have a multi-digit number **
			else:			# If the element has already been counted
				thiscompound[Existence[1]] = inp[i]		# Put this number in the appropriate place on the list.
    return [elements, thiscompound]		# Return the updated elements list and the compounds vector

# # # # # # # # # # # # # # # # # # # # # # # # # # # # # # # # # # # # # # # # # # # # # # #

equation = input("Please enter the chemical equation: \nUse no spaces. Use '=' for the arrow. Do not exclude 1 as a subscript. Simplify all compounds.")			# Asks user to enter equation
numElements = int(input("Count the total number of elements in the equation \n For example CHO+O2=CO2+H20 has 3 elements, C H and O."))			# Asks user to enter number of elements in equation
globalElements = []	# A list to keep track of which elements are which entry in each vector

sides = split(equation, '=')			# Splits the equation into the reactants and the products
if len(sides) != 2:		# Test if there is more than one product string and one reactant string
	print("!!!Product/reactant split error!!!")				# Print error message is this happens

reactants = split(sides[0],'+') 	# Takes first entry of sides[] and splits it, these are the reactants
vector = [] 		# An empty vector to use in the following loop
reactantVectors = [] # This will be a list of lists to keep track of all the vectors for the reactants

# Creates an empty vector for each reactant and each vector is the appropriate size for the number of elements.
for i in reactants:  			# For each entry in list reactants
	vector = []  					# Zero out the vector
	for i in range(numElements):	 # Loop ``the number of elements, numElements'' times
		vector.append(0)     					# Add a zero to the vector
	reactantVectors.append(vector) 		# Add that vector to our reactant Vectors

for i in range(len(reactants)):  # Loop ``for how many items in list reactant'' times
	# Vectorize the reactant. Inputs the reactant string, the corresponding empty vector, and the list of elements
	vectorization = vectorizeCompound(reactants[i], reactantVectors[i], globalElements)
	globalElements = vectorization[0]  	# Update element list
	reactantVectors[i] = vectorization[1]		# Update the compound vectors.

# The following section is the exact same is above, just for the products of the equation
products = split(sides[1],'+')  	# Takes second entry of sides[] and splits it, these are the products
vector = []
productVectors = []
for i in products:
	vector = []
	for i in range(numElements):
		vector.append(0)
	productVectors.append(vector)
  
for i in range(len(products)):
	vectorization = vectorizeCompound(products[i], productVectors[i], globalElements)
	globalElements = vectorization[0]
	productVectors[i] = vectorization[1]
 
 # Multiplies all entries of the product vectors by -1 to simulate when we moved them across the equals sign.
for i in range(len(productVectors)):			# Loop through only the product vectors
	for c in range(len(productVectors[i])):		# Loop through all the entries in each vector
		productVectors[i][c] = -1 * int(productVectors[i][c])			# Multiply them all by -1
    
equationMatrix = reactantVectors # Create a new matrix (list of lists) that will store both reactant vectors and product vectors. This starts the list by copying the reactantVectors
for i in productVectors:     # For all the rows in the product matrix, add them to the matrix
	equationMatrix.append(i)
  
E = Matrix(equationMatrix)   # Converts the list of lists into an sympy matrix
E = E.transpose()        # Transpose the matrix so that the now row vectors become column vectors
A = E.rref()          # Row reduce this matrix and call it A

# # # # # # # # # # # # # # # # # # # # # # # # # # # # # # # # # # # # # # # # # # # # # # #

FreeVariables = []    	# Create a list for the free variables
for i in equationMatrix:   # This is the list of lists that became the matrix E. Since E got transposed, the number of items in this variable is the number of columns in the matrix E
	FreeVariables.append(1)  # Create an entry in the free variables list for every column of the matrix

for i in range(len(E.rref()[1])): # Loops ``for number of columns of E with pivots in them'' times
	FreeVariables[E.rref()[1][i]] = 0 # 'Turn off' each of our columns that is not associated with a free variable

Entries = [] 		# Save the value for what we should choose our free variables to be, so every output is a whole number
for c in range(len(FreeVariables)):		# Iterate through our free variables list
	if FreeVariables[c] == 1:			# If this column represents a free variable
		for r in range(E.shape[0]):		# Iterate through the entries of that column
			if A[0][r,c] != 0:			# As long as the entry is not zero
				Entries.append(Fraction(str(A[0][r,c])).denominator)  		# Then save the denominator of the fraction
# The str(-sympy rational-) is neccesary because the float makes the denominator innacurate -- it will evaluate 1/3 as 3333333/10000000  

# Finds the least common multiple of two numbers
def lcm(a,b):       
	return (a*b) / math.gcd(a,b)  	# The product of the two numbers, divided by the greatest common divisor of those two.


Current = Entries[0]  # This is just an initial condition to avoid out of range index errors **
for i in range(len(Entries)): # Iterate through our denominator list
	if i == 0:    # Forget the first entry, we already set that up above **
		continue  
	Current = lcm(int(Current), int(Entries[i])) # Find the least common multiple of the current least common multiple and the current entry in the list.

V = [] 		# A vector to multiply our rows by to get to the solution.
for i in FreeVariables:  		# Iterate through our free variables list
	if i == 1:     						# If the column is associated with a free variable
		V.append(Current) 			# Multiply it by the largest multiple
	else:         						# If the column is a pivot column
		V.append(0)   			 		# Zero the column out
		# This way, when we add the rows together we will get what the other variables equal.


F = A[0] * Matrix(V) 		# Do this multiplication to get our answer as F

Coefficients = []      		# An output medium for the solution     
for i in range(len(FreeVariables)):  	# Iterate through our columns
	if FreeVariables[i] == 1:    	 # If it represents our free variable
		Coefficients.append(int(Current))		# Make this free variable our chosen amount that makes all the numbers whole
	else: 			# If the column is not a free variable
		Coefficients.append(-1 * int(F[i])) # Just take that entry from our solution vector, and invert it.
print(equation)  			# Print the equation we set out to solve
print(Coefficients) 	# and the coefficients we have solved for.
print("These are the coefficients in order that the compounds occur.")  # Hopefully it works.
\end{lstlisting}

\section{Conclusion}
%For conclusion, just reiterate all this stuff, ``now that they've read it'' and just talk about what we could do differently
%or what could we do more.
It is amazing to see how a process such as balancing a chemical equation can seem so simple to do by hand but then look so
complicated when being coded for a program. This is because we need to write very specific instructions for the computer to
follow in order to do such a task. In doing this, we have created a much faster and more efficient way of balancing chemical
equations than by doing them by hand. This would then save people time and money. However, this code can be improved in
many ways and made more efficient. For example, it could be altered so that the user does not have to enter the equation
without spaces or so that they would not have to enter a 1 for atoms without subscripts. Future implications of this program
could be adding in some sort of GUI for the user to interact with or adding graphics so they can visually see the process of
balancing the chemical equation. In reading this, I hope you have gained a better understanding of the process of balancing a
chemical equation using linear algebra and the use of programming and technology to improve it.

\end{document}